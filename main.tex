\documentclass[english,oneside]{pucrs-ppgcc}
%\documentclass[english,twoside]{pucrs-ppgcc}
% \RequirePackage[T1]{fontenc}[2005/09/27]
% \RequirePackage[utf8x]{inputenc}[2008/03/30]
% \RequirePackage[english,brazil]{babel}[2008/07/06]
% \RequirePackage[a4paper]{geometry}[2010/09/12]
% \RequirePackage{textcomp}[2005/09/27]
% \RequirePackage{lmodern}[2009/10/30]
% \RequirePackage{indentfirst}[1995/11/23]
% \RequirePackage{setspace}[2000/12/01]
% \RequirePackage{textcase}[2004/10/07]
% \RequirePackage{float}[2001/11/08]
% \RequirePackage{amsmath}[2000/07/18]
% \RequirePackage{amssymb}[2009/06/22]
% \RequirePackage{amsfonts}[2009/06/22]
% \RequirePackage{url}
% \RequirePackage[table]{xcolor}[2007/01/21]
%\RequirePackage{array}[2008/09/09]
%\RequirePackage{longtable}
\usepackage{graphicx}
% Utilize a opção 'pdftex' se você estiver usando o pdflatex (que
% permite figuras em formatos como .jpg ou .png)
%\usepackage[pdftex]{graphicx}
\usepackage{multirow}
\usepackage{nicefrac}
% Para inserir figuras lado a lado.
% \usepackage{subfigure}
% Para formatar algoritmos.
% A opção [algo2e] é necessária para evitar conflitos
% com as definições da classe.
%\usepackage[algo2e]{algorithm2e}
\usepackage{algorithmic}
% Um float do tipo algoritmo. No momento
% este pacote é incompatível com a classe.
%\usepackage{algorithm}
\usepackage{bookmark}
\usepackage{import}
\usepackage{todonotes}
\usepackage{enumitem}
\usepackage[table]{xcolor}
\usepackage{tabularx}
\setcitestyle{square}
\usepackage{lscape}
\usepackage{listings}

\usepackage{caption}

\newcommand{\levelA}{\chapter}
\newcommand{\levelB}{\section}
\newcommand{\levelC}{\subsection}

\author{Atila Leites Romero}
\title{Redes neurais aplicadas a classificação de fragmentos de arquivos}
      {Applied neural networks for file fragment classification}

%----------------------------------------------------------------
% Opções para o tipo de trabalho (OBRIGATÓRIO)
%----------------------------------------------------------------
%\tipotrabalho{\monografia}  % Monografias em geral (e de "bônus": TCCs)
%\tipotrabalho{\pep}         % Plano de estudo e pesquisa
\tipotrabalho{\dissertacao} % Dissertação
%\tipotrabalho{\ptese}       % Proposta de tese
%\tipotrabalho{\tese}        % Tese

%----------------------------------------------------------------
% Seleção do curso ("este trabalho é um requisito parcial para
% obtenção do grau de (mestre ou doutor) em Ciência da Computação").
% Necessário somente para o tipo 'monografia'.
%----------------------------------------------------------------
%\grau{\bacharel} % Este é "bônus"
\grau{\mestre}
%\grau{\doutor}

%----------------------------------------------------------------
% Orientador (e Co-orientador, caso haja um). É OBRIGATÓRIO
% informar pelo menos o orientador.
%----------------------------------------------------------------
\orientador{Dr. Avelino Francisco Zorzo}
%\coorientador{Ciclano de Farias}

%----------------------------------------------------------------
% A capa é inserida automaticamente. Por isso não é necessário
% chamar \maketitle
%----------------------------------------------------------------
\begin{document}

%----------------------------------------------------------------
% Depois da capa vem a dedicatória e a epígrafe.
%----------------------------------------------------------------

%\dedicatoria{Dedico este trabalho à minha família.}
%
\epigrafe{If we are offered several hypotheses, we should begin our considerations by striking the most complex of them with our sword.}
         {Isaac Asimov and Robert Silverberg - \cite{asimov_nightfall_2011}}

%----------------------------------------------------------------
% Também dá para fazer as duas na mesma página:
%----------------------------------------------------------------
% \dedigrafe{Dedico este trabalho à minha família.}
%           {If we are offered several hypotheses, we should begin our considerations by striking the most complex of them with our sword.}
%           {Isaac Asimov}

%----------------------------------------------------------------
% A seguir, a página de agradecimentos (OPCIONAL):
%----------------------------------------------------------------
\begin{agradecimentos}
%O presente trabalho foi realizado com apoio da Coordenação de Aperfeiçoamento de Pessoal Nivel Superior – Brasil (CAPES) – Código de Financiamento 001
This study was financed in part by the Coordenação de Aperfeiçoamento de Pessoal de Nivel Superior – Brasil (CAPES) – Finance Code 001.

I would also like to acknowledge the support provided by the Pontifical Catholic University of Rio Grande do Sul (PUCRS) and by the National Institute of Science and Technology on Forensic Sciences (INCT) during the preparation of this dissertation.

\end{agradecimentos}

%----------------------------------------------------------------
% Resumo, com as palavras-chave passadas por parâmetro
% (OBRIGATÓRIO, ao menos para teses e dissertações)
%----------------------------------------------------------------
\begin{resumo}{computação forense, \textit{data carving}, aprendizado de máquina, redes neurais, Long Short-Term Memory, camadas convolucionais, classificação de fragmentos de arquivo, dados de alta entropia}

Este trabalho está dividido em três partes.
Na primeira, os valores de acurácia de alguns tipos de redes neurais são comparados, classificando fragmentos de arquivo extraídos do \textit{dataset} Govdocs1 \cite{garfinkel_bringing_2009}, usando suas extensões de arquivo como classes.
Os resultados sugerem que os tipos de arquivo que compõem o \textit{dataset} têm um papel mais relevante nos resultados do que os detalhes de arquitetura dos modelos de redes neurais.
Na segunda parte, a influência do número de classes na acurácia é explorada.
A conclusão é que o número de classes no \textit{dataset} é relevante, mas menos importante que os tipos de extensões selecionadas para compô-lo.
Finalmente, um procedimento é criado para testar a hipótese de que parte dos erros de classificação de fragmentos de arquivos pode ser explicada pela incapacidade dos modelos de distinguir dados de alta entropia de dados aleatórios.
A conclusão é que, para o modelo usado, a incapacidade de distinguir dados de alta entropia de dados aleatórios pode explicar apenas 1/3 dos erros.
Isso sugere que, para pesquisas futuras sobre classificação de fragmentos de arquivos, a busca por um procedimento para rotular automaticamente estruturas internas de arquivos pode ser uma abordagem promissora.

\end{resumo}





%----------------------------------------------------------------
% Abstract, com as palavras-chave passadas por parâmetro
% (OBRIGATÓRIO, ao menos para teses e dissertações)
%----------------------------------------------------------------
\begin{abstract}{computer forensics, data carving, machine learning, neural networks, long short-term memory, convolutional layers, file fragment classification, high entropy data}

High entropy data, frequently found on images, video, and compressed files, may be misinterpreted as random data.
This research explores how far the presence of high entropy data can explain file fragment classification errors. The research is divided into three parts. In the first, the accuracy of some types of neural networks is compared, classifying file fragments taken from the Govdocs1 dataset\cite{garfinkel_bringing_2009}, using their file extensions as class labels. Then, the influence of the number of classes on the accuracy of the resulting models is explored. Finally, an experiment is devised to measure, for a given neural network architecture, the number of fragments that have recognizable structures for each file type. The conclusion is that, for the model used, which classifies fragments of 28 file types, misinterpretation of high entropy data as random data can only account for an error in accuracy of 16.5\%. Since the total error in accuracy is 46\% \todo{revise later} for 28 classes, this leaves 29.5\% of samples misclassified by other reasons. This suggests that, for future researches on file fragment classification, the seek for a procedure to automatically label inner data structures of files may be a promising approach.

\end{abstract}

\setcounter{tocdepth}{1}
%----------------------------------------------------------------
% Listas e sumário, nessa ordem. Somente o sumário é obrigatório,
% portanto, comente as outras listas, caso sejam desnecessárias.
%----------------------------------------------------------------
\listoffigures       % Lista de figuras      (OPCIONAL)
\listoftables        % Lista de tabelas      (OPCIONAL)
\listofalgorithms    % Lista de algoritmos   (OPCIONAL)
\listofacronyms      % Lista de siglas       (OPCIONAL)
\listofabbreviations % Lista de abreviaturas (OPCIONAL)
\listofsymbols       % Lista de símbolos     (OPCIONAL)
\tableofcontents     % Sumário               (OBRIGATÓRIO)

%----------------------------------------------------------------
% Aqui começa o desenvolvimento do trabalho. Para uma melhor
% organização do documento, separe-o em arquivos,
% um para cada capítulo. Para isso, utilize o comando \include,
% como mostrado abaixo.
%----------------------------------------------------------------
\subimport{content/}{0-sections}

%----------------------------------------------------------------
% Aqui vai a bibliografia. Existem dois estilos de citação: use
% 'ppgcc-alpha' para citações do tipo [Abc+] ou [XYZ] (em ordem
% alfabética na bibliografia), e 'ppgcc-num' para citações
% numéricas do tipo [1], [20], etc., em ordem de referência.
%----------------------------------------------------------------
\bibliographystyle{ppgcc-alpha}
%\bibliographystyle{ppgcc-num}
%\bibliographystyle{apalike}
\bibliography{zotero}

%----------------------------------------------------------------
% Após \appendix, se iniciam os capítulos de Apêndice, com
% numeração alfabética.
%----------------------------------------------------------------

%----------------------------------------------------------------
% Aqui vão os "capítulos" de anexos. Cada anexo deve
% ser considerado um capítulo.
%----------------------------------------------------------------

% E aqui (para a felicidade de todos) termina o documento.
\end{document}
