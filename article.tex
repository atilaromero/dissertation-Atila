\documentclass[english,oneside]{article}
% \documentclass[english,oneside,twocolumn]{article}
\RequirePackage[T1]{fontenc}
\RequirePackage[utf8x]{inputenc}
\RequirePackage[english]{babel}
\RequirePackage[a4paper]{geometry}
\RequirePackage{textcomp}
\RequirePackage{lmodern}
\RequirePackage{indentfirst}
\RequirePackage{setspace}
\RequirePackage{textcase}
\RequirePackage{float}
\RequirePackage{amsmath}
\RequirePackage{amssymb}
\RequirePackage{amsfonts}
\RequirePackage{url}
\RequirePackage[table]{xcolor}
\RequirePackage{array}
\RequirePackage{longtable}
\usepackage{graphicx}
% Utilize a opção 'pdftex' se você estiver usando o pdflatex (que
% permite figuras em formatos como .jpg ou .png)
%\usepackage[pdftex]{graphicx}
\usepackage{multirow}
\usepackage{nicefrac}
% Para inserir figuras lado a lado.
% \usepackage{subfigure}
% Para formatar algoritmos.
% A opção [algo2e] é necessária para evitar conflitos
% com as definições da classe.
%\usepackage[algo2e]{algorithm2e}
\usepackage{algorithmic}
% Um float do tipo algoritmo. No momento
% este pacote é incompatível com a classe.
\usepackage{algorithm}
\usepackage{bookmark}
\usepackage{import}
\usepackage{todonotes}
\usepackage{enumitem}
\usepackage[table]{xcolor}
\usepackage{tabularx}
% \setcitestyle{square}
\usepackage{lscape}
\usepackage{listings}

\usepackage{caption}

\newcommand{\sigla}[2]{}
\newcommand{\levelA}{\section}
\newcommand{\levelB}{\subsection}
\newcommand{\levelC}{\subsubsection}

\author{Atila L. Romero\textsuperscript{$\dagger$}, Avelino F. Zorzo\textsuperscript{$\dagger$}{\let\thefootnote\relax\footnote{{\textsuperscript{$\dagger$}e-mails: \texttt{atila.romero@acad.pucrs.br, avelino.zorzo@pucrs.br}}}}}
\title{Data carving using neural networks}
\date{}

\begin{document}
\maketitle


\begin{abstract}{computer forensics, data carving, machine learning, neural networks, long short-term memory, convolutional layers, file fragment classification, high entropy data}

This work is divided in three parts. 
In the first, the accuracy values of some types of neural networks are compared, classifying file fragments taken from the Govdocs1 dataset\cite{garfinkel_bringing_2009}, using their file extensions as class labels.
The results suggest that the file types composing the dataset has a more relevant role in the results than the architectural details of the neural network models.
In the second part, the influence of number of classes on accuracy is explored. 
The conclusion is that the number of classes in the dataset is relevant but less important than the types of extensions selected to compose the dataset.
Finally, a procedure is devised to test the hypothesis that part of the errors of file fragment classification can be explained by the inability of the models to distinguish high entropy data from random data.
The conclusion is that, for the model used, the inability to distinguish high entropy data from random data can only account for about 1/3 of the errors.
This suggests that, for future researches on file fragment classification, the seek for a procedure to automatically label inner data structures of files may be a promising approach.

\end{abstract}


\subimport{content/}{0-sections}

%----------------------------------------------------------------
% Aqui vai a bibliografia. Existem dois estilos de citação: use
% 'ppgcc-alpha' para citações do tipo [Abc+] ou [XYZ] (em ordem
% alfabética na bibliografia), e 'ppgcc-num' para citações
% numéricas do tipo [1], [20], etc., em ordem de referência.
%----------------------------------------------------------------
% \bibliographystyle{ppgcc-alpha}
%\bibliographystyle{ppgcc-num}
\bibliographystyle{plain}
\bibliography{zotero}

\end{document}
