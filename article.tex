\documentclass[english,oneside]{article}
\RequirePackage[T1]{fontenc}
\RequirePackage[utf8x]{inputenc}
\RequirePackage[english]{babel}
\RequirePackage[a4paper]{geometry}
\RequirePackage{textcomp}
\RequirePackage{lmodern}
\RequirePackage{indentfirst}
\RequirePackage{setspace}
\RequirePackage{textcase}
\RequirePackage{float}
\RequirePackage{amsmath}
\RequirePackage{amssymb}
\RequirePackage{amsfonts}
\RequirePackage{url}
\RequirePackage[table]{xcolor}
\RequirePackage{array}
\RequirePackage{longtable}
\usepackage{graphicx}
% Utilize a opção 'pdftex' se você estiver usando o pdflatex (que
% permite figuras em formatos como .jpg ou .png)
%\usepackage[pdftex]{graphicx}
\usepackage{multirow}
\usepackage{nicefrac}
% Para inserir figuras lado a lado.
% \usepackage{subfigure}
% Para formatar algoritmos.
% A opção [algo2e] é necessária para evitar conflitos
% com as definições da classe.
%\usepackage[algo2e]{algorithm2e}
\usepackage{algorithmic}
% Um float do tipo algoritmo. No momento
% este pacote é incompatível com a classe.
\usepackage{algorithm}
\usepackage{bookmark}
\usepackage{import}
\usepackage{todonotes}
\usepackage{enumitem}
\usepackage[table]{xcolor}
\usepackage{tabularx}
% \setcitestyle{square}
\usepackage{lscape}
\usepackage{listings}

\usepackage{caption}

\newcommand{\sigla}[2]{}
\newcommand{\levelA}{\section}
\newcommand{\levelB}{\subsection}
\newcommand{\levelC}{\subsubsection}

\author{Atila L. Romero\textsuperscript{$\dagger$}, Avelino F. Zorzo\textsuperscript{$\dagger$}{\let\thefootnote\relax\footnote{{\textsuperscript{$\dagger$}e-mails: \texttt{atila.romero@acad.pucrs.br, avelino.zorzo@pucrs.br}}}}}
\title{Data carving using neural networks}
\date{}

\begin{document}
\maketitle

\begin{abstract}
% explorations
% - lstm on data carving
% - automatization of new data carving parsers
% - pratical solution
%from pep 1, paragrafo 7 e pep abstract, paragrafo1
This work explores the use of neural networks to perform data carving, comparing some artificial neural network architectures. The considered networks were trained without special hardware in short periods of time using limited datasets. It is expected that neural networks can be used to automate the construction of data carving solutions, allowing forensic examiners to collaboratively build models for several file types.

\textbf{Keywords:} computer forensics, carving, machine learning, neural networks, long short-term memory, convolutional layers
\end{abstract}

\subimport{content/}{0-sections}

%----------------------------------------------------------------
% Aqui vai a bibliografia. Existem dois estilos de citação: use
% 'ppgcc-alpha' para citações do tipo [Abc+] ou [XYZ] (em ordem
% alfabética na bibliografia), e 'ppgcc-num' para citações
% numéricas do tipo [1], [20], etc., em ordem de referência.
%----------------------------------------------------------------
% \bibliographystyle{ppgcc-alpha}
%\bibliographystyle{ppgcc-num}
\bibliographystyle{plain}
\bibliography{zotero}

\end{document}
