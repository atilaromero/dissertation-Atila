%\levelB{Models}
The networks under consideration used different combinations of convolutional, max pooling, LSTM, and fully connected layers,
%loss
applying categorical cross-entropy as the loss function. Binary cross-entropy and mean squared error were considered during initial tests, but categorical cross-entropy gave faster training times.

Fourteen models participated in this evaluation. One of then is a simple single-layer perceptron. Two of them use LSTM layers without convolutional layers, while 3 of them use convolutional layers without LSTM layers. Eight models combine convolutional layers and LSTM layers. Table \ref{tab:models} outline the parameters of the models, which can be analyzed in more detail in the code repository \sloppy\url{http://github.com/atilaromero/carving-experiments}. The convolutional layers do not use padding. 

\newcommand{\paint}{\cellcolor{gray!25}}
\begin{table}[!ht]
    \centering
    \footnotesize
    \caption{14 models trained on 30 classes}
    \label{tab:models}
\begin{tabular}{l||l|l|l||l|l|l||l|l||l}
\hline
Model  & \multicolumn{3}{l||}{Convolutional layers} & \multicolumn{3}{l||}{Max pooling} & \multicolumn{2}{l||}{LSTM} & Dense  \\ \hline
       & output       & window       & stride      & pool    & stride    & channel    & output     & return       & output \\ 
       & size         & size         & length      & size    & length    & first      & size       & sequences    & size   \\ \hline
                                                                                                                              \hline
D      & \paint       & \paint       &     \paint  & \paint  &  \paint   & \paint     & \paint     & \paint       & 30     \\ \hline
LD     & \paint       & \paint       &     \paint  & \paint  &  \paint   & \paint     & 32         & no           & 30     \\ \hline
LL     & \paint       & \paint       &     \paint  & \paint  &  \paint   & \paint     & 32         & yes          & \paint \\ 
       & \paint       & \paint       &     \paint  & \paint  &  \paint   & \paint     & 32         & no           & \paint \\ \hline
CD     & 64           & 64           & 8           & \paint  &  \paint   & \paint     & \paint     & \paint       & 30     \\ \hline
CM     & 30           & 32           & 1           & 481     & 1         & no         &  \paint    & \paint       & \paint \\ \hline
CCM    & 30           & 32           & 1           &         &           &            & \paint     & \paint       & \paint \\ 
       & 30           & 2            & 2           & 240     & 1         & no         & \paint     & \paint       & \paint \\ \hline
CL     & 30           & 32           & 32          & \paint  &  \paint   & \paint     & 30         & no           & \paint \\ \hline
CML    & 32           & 32           & 32          & 2       & 2         & yes        & 30         & no           & \paint \\ \hline
CLL    & 32           & 32           & 32          & \paint  &  \paint   & \paint     & 64         & yes          & \paint \\ 
       &              &              &             & \paint  &  \paint   & \paint     & 30         & no           & \paint \\ \hline
CLD    & 256          & 16           & 16          & \paint  &  \paint   & \paint     & 128        & no           & 30     \\ \hline
CCL    & 128          & 8            & 8           & \paint  &  \paint   & \paint     &            &              & \paint \\ 
       & 64           & 8            & 8           & \paint  &  \paint   & \paint     & 30         & no           & \paint \\ \hline
CCLL   & 128          & 8            & 8           & \paint  &  \paint   & \paint     & 64         & yes          & \paint \\ 
       & 64           & 8            & 8           & \paint  &  \paint   & \paint     & 30         & no           & \paint \\ \hline
CMCML  & 128          & 8            & 8           & 2       & 2         & yes        &            &              & \paint \\ 
       & 64           & 8            & 8           & 2       & 2         & yes        & 30         & no           & \paint \\ \hline
CMCMLL & 128          & 8            & 8           & 2       & 2         & yes        & 64         & yes          & \paint \\ 
       & 64           & 8            & 8           & 2       & 2         & yes        & 30         & no           & \paint \\ \hline
\end{tabular}
\end{table}

%optimization
All the trainings used the Adam \cite{kingma_adam:_2014}
optimization algorithm to guide backpropagation, which was selected because it performed well in the preliminary tests without requiring fine-tuning of parameters.

% Constraints
The models were trained until no further improvement was observed in the last 10 epochs, using categorical cross entropy loss.

After the 14 models were compared, one of the models, CLD, was chosen to have its results compared with other works. A longer training session was performed in this stage, using a batch size of 300 samples instead of 100, and increasing the improvement threshold of the early stopping condition (maximum number of epochs to wait for improvements) from 10 to 20.

Using this extended training parameters, the CLD model was trained with the 28 file types used in the first stage. Then, a different model was built from scratch using the same file types of each of the most recent works listed in table \ref{tab:datacarvingstudies}: 
Hiester \cite{hiester_file_2018}, 
Chen \textit{et al.} \cite{chen_file_2018},
Wang \textit{et al.} \cite{wang_sparse_2018},
Wang \textit{et al.} \cite{wang_file_2018},
and
Vulinović \textit{et al.} \cite{vulinovic_neural_2019}.

This comparison has three important restrictions.
Chen used 4096 bytes for the block size, while this work used 512 bytes only.
Vulinović got an F1-score of 81\% for the Multi Layer Perceptron (MLP) feeded with byte histograms, a better result than the one obtained with a Convolutional Neural Network (CNN) feeded with raw bytes, 61\%. But the comparison with the CLD model of this study was based on Vulinović results with the CNN, not the MLP, because it is a more similar solution, one that uses raw bytes instead of byte histograms. Additionally, his results use F1-score, while this work measured accuracy. 

