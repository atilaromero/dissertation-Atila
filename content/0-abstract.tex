\begin{abstract}
% explorations
% - lstm on data carving
% - automatization of new data carving parsers
% - pratical solution
%from pep 1, paragrafo 7 e pep abstract, paragrafo1
High entropy data, frequently found on images, video, and compressed files, may be misinterpreted as random data.
This research explores how far the presence of high entropy data can explain classification errors at the file fragment classification task. The research is divided into three parts. In the first, the accuracy of some types of neural networks is compared, classifying file fragments taken from the Govdocs1 dataset\cite{garfinkel_bringing_2009}, using their file extensions as class labels. Then, the influence of the number of classes on the accuracy of the resulting models is briefly explored. Finally, an experiment is devised to measure, for a given neural network architecture, the number of fragments that have recognizable structures for each file type. The conclusion is that, for the model used, which classifies fragments of 28 file types, misinterpretation of high entropy data as random data can only account for an error in accuracy of 16.5\%. Since the total error in accuracy is 46\% \todo{revise later} for 28 classes, this leaves 29.5\% of samples misclassified by other reasons. This suggests that, for future researches on file fragment classification, the seek for a procedure to automatically label inner data structures of files may be a promising course of action.


% This work explores the use of neural networks to perform data carving, comparing some artificial neural network architectures. The considered networks were trained without special hardware in short periods of time using limited datasets. It is expected that neural networks can be used to automate the construction of data carving solutions, allowing forensic examiners to collaboratively build models for several file types.
\textbf{Keywords:} computer forensics, data carving, machine learning, neural networks, long short-term memory, convolutional layers, file fragment classification, high entropy data
\end{abstract}
