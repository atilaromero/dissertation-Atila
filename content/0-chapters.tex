%!TEX root = ../main.tex
\sigla{LSTM}{Long Short-Term Memory}
% \abrev{Abrev}{Abreviatura}
% \simbolo{Hz}{Hertz}

\listoftodos

\chapter{\label{chap:introduction}Introduction}
\todo[inline]{chapter, figure, table, section: to uppercase}

%\section{Motivation}
\section{Motivation}
% establishing the context, background and/or importance of the topic
    
% file recovery: motivation
% data carving: sometimes the only solution
%from pep 1, paragrafo 1
In a forensic context, file recovery is a frequent task that can be motivated by several situations, like physical media malfunction, intentional attempt to hide data, and the need to access deleted or older versions of files. When the filesystem no longer provides the physical location of a file on the media, data carving is often the only procedure capable of retrieving its content.


% the problem of data carving development
%from pep 1, paragrafo 4
The patterns searched by data carving software are generally manually coded, taking advantage of fixed byte sequences found on headers and footers. But the amount of different file types combined with the slow process of manually coding each of those patterns makes the development of data carving software a tedious task \cite{mcdaniel_content_2003}.

% ml as a solution
%from pep 1, paragrafo 5
The application of machine learning solutions to this manual task has the potential to make it easier and faster. An initial strategy could be to train a classifier to, given a chunk of data, provide a label indicating a file type. That could be used to recover unfragmented deleted files.
% novo
Then, that same classifier can be applied in small chunks of data to produce input to a second algorithm, responsible to reassemble the fragments of a file.

% ml as a solution
%from pep 1, paragrafo 6
% The recovery of fragmented files through data carving would require some sort of pattern recognition on the identified chunks, in order to reconstruct the correct sequence.

% \section{Relevant literature}
% \section{Relevant literature}
% giving a brief synopsis of the relevant literature

% \todo[inline]{summary of literature}
% \todo[inline]{deficiencies in literature}

%\section{Research questions}
\section{Research questions}
% listing the research questions or hypotheses

% main objective: lstm on data carving
%from pep 4.1, paragrafo 1
The major objective of this study is to advance the research on the use of Long Short-Term Memory (LSTM) neural networks to perform data carving, answering the following initial question:

%from pep 4.1
\begin{enumerate}[itemindent=\parindent,label=\textbf{Q\arabic*.}]

    \item How does LSTM compare to other techniques in terms of training performance and quality of results?
    
    \item Which LSTM-based models would be more suitable to accept the addition of new file types by the forensic examiners community? 

\end{enumerate}

%\section{Overview}
\section{Overview}
% \subsection{Outline}
% providing an overview of the dissertation or report structure

The remainder of this document is organized as follows.
\todo[inline]{check document outline}
    Chapter 2 analyses the current status of data carving tools and research. 
    Chapter 3 analyses how current research on sequence labeling can improve data carving solutions.
    Chapter 4 proposes solutions to some of the data carving problems.
    Chapter 5, for each performed experiment, describes the chosen method, presents the results obtained and offers an discussion of the results.
    Chapter 6 summarizes the work, analysing achievements and limitations, and including suggestions for future work.


\chapter{\label{chap:background}Background}
\todo[inline]{introduction describing the rest of the chapter}
\todo[inline]{define: backpropagation}
\todo[inline]{define: forward propagation}
\todo[inline]{define: RNN}
\todo[inline]{define: feedforward}
\todo[inline]{define: tipos de nn}
\todo[inline]{define: pooling layer}
\todo[inline]{define: stride}
\todo[inline]{define: slide}
\todo[inline]{define: kernel}
\todo[inline]{define: translation invariance}
\todo[inline]{define: categorical cross-entropy}
\todo[inline]{define: padding}
\todo[inline]{define: epochs}
\todo[inline]{define: improvement threshold of early stopping condition}
\todo[inline]{define: softmax}

\section{\label{sec:datacarving}Data carving}
\section{\label{sec:datacarving}Data carving}
% \subsection{Description}
%from pep 1, paragrafo 2
Data carving is a forensic process that attempts to recover files without previous information of where the file starts or ends \cite{garfinkel_carving_2007}.
To accomplish this, a software has to analyze a source of raw data, searching for patterns indicating a known file type and making attempts to locate and reconstruct each of its constituent parts.

%from pep 1, paragrafo 2b
That process commonly disregards the filesystem \cite{veenman_statistical_2007}, being able to recover deleted files from unallocated areas, but faces the problem of fragmentation \cite{veenman_statistical_2007}  \cite{pal_evolution_2009}: in many cases, files are not written sequentially on disk and deleted files may have missing parts.

This procedure is frequently used in forensic environments, but it may also be beneficial in other areas, such as reverse engineering, network traffic analysis, and data mining.

This observation is related to the fact that many types of data sources contains embedded files. Therefore, they may be used as input to a data carving process. This includes network traffic, memory dumps, hard drive images, and files containing other files.

To show how a data carving process work, a hard drive with a deleted video file is here used as an example.
Normally, the average user of a computer does not need to deal with hard disk sectors directly and has contact only with the already mounted filesystem, which presents directories and files for the user. But in order to present such view, the operating system has to interpret the raw hard drive data, which is simply a stream of data blocks. The first blocks of the drive contain a partition table, indicating ranges of blocks belonging to each of the drive partitions. Inside a partition, the operating system expects to find a filesystem. The filesystem stores metadata about each file or directory and keeps an index indicating the position of each file on disk. This way, when a user opens a file, the operating system uses the filesystem to find where in the disk the file is stored, accessing those areas directly and returning its content to the user, who sees the returned data as the file content. 
 
If video file in the example had not been deleted, the filesystem could use the index to find the video location on disk. But after deletion, the corresponding index entry is erased, while the actual content of the file may be left untouched to avoid disk access. In this circumstance, a data carving procedure may successfully retrieve the file, even if the filesystem cannot. One common data carving approach that does not deal with fragmentation consists in searching for headers and footers patterns. To retrieve the hypothetical video file in the example, a software using this approach would sequentially read each drive sector, find a known header of a video file and save the following sectors until a footer is found or a size limit is reached. 
 
\input{content/2.1.1-evolution.tex}

\input{content/2.1.2-tools.tex}

\input{content/2.1.3-nn.tex}


\section{\label{sec:sequencelabeling}Sequence labeling}

\section{\label{sec:lstm}Long Short-Term Memory}
\input{content/2.2.1-lstm.tex}

\chapter{Related work}
\chapter{Related work}
\input{content/3.1-datacarvingevolution.tex}
%from pep 4, paragrafo 2
Several authors reviewed  available data carving tools
\cite{ali_review_2018}
\cite{qiu_new_2014}
\cite{nadeem_ashraf_forensic_2013}
\cite{roux_reconstructing_2008}, 
but the tool listing from Ali \textit{et al.} \cite{ali_review_2018} was found to be the most comprehensive one. Among the listed tools, only Foremost \cite{kendall_foremost_2019}, Scalpel \cite{richard_iii_scalpel:_2005}, and PhotoRec \cite{grenier_photorec_2019} support a wide range of file formats. For example, Photorec supports more than 300 file types.

%from pep 4, paragrafo 1
The available data carving tools generally do not take advantage of the latest techniques that research on the field offers, often still relying on header/footer identification and providing limited reassembling capabilities.
%from pep 2, paragrafo 11
According to Ali \textit{et al.} \cite{ali_review_2018}, artificial intelligence techniques are found to be not fully utilized in this field.

Comparing the accuracy between research papers and available software is difficult because, as they rely on header and footer identification, their performance would be more properly compared to whole file classification instead of file fragment classification, the latter being a harder problem.

\input{content/3.4-challenges.tex}
\input{content/3.3-nn.tex}


    \section{Data carving evolution}
    \input{content/3.1-datacarvingevolution.tex}
    \section{Current data carving tools}
    %from pep 4, paragrafo 2
Several authors reviewed  available data carving tools
\cite{ali_review_2018}
\cite{qiu_new_2014}
\cite{nadeem_ashraf_forensic_2013}
\cite{roux_reconstructing_2008}, 
but the tool listing from Ali \textit{et al.} \cite{ali_review_2018} was found to be the most comprehensive one. Among the listed tools, only Foremost \cite{kendall_foremost_2019}, Scalpel \cite{richard_iii_scalpel:_2005}, and PhotoRec \cite{grenier_photorec_2019} support a wide range of file formats. For example, Photorec supports more than 300 file types.

%from pep 4, paragrafo 1
The available data carving tools generally do not take advantage of the latest techniques that research on the field offers, often still relying on header/footer identification and providing limited reassembling capabilities.
%from pep 2, paragrafo 11
According to Ali \textit{et al.} \cite{ali_review_2018}, artificial intelligence techniques are found to be not fully utilized in this field.

Comparing the accuracy between research papers and available software is difficult because, as they rely on header and footer identification, their performance would be more properly compared to whole file classification instead of file fragment classification, the latter being a harder problem.

    \section{Data carving challenges}
    \input{content/3.4-challenges.tex}
    \section{Neural networks research in data carving}
    \input{content/3.3-nn.tex}
% \input{content/4-researchmethods.tex}
% \input{content/4-environment-tools.tex}
% \section{Environment validation procedures}
Before starting to experiment directly with different data carving models,
some tests dealing with other more popular problems were performed.
The goal of this approach was to test the Keras and TensorFlow environment, to learn machine learning techniques using these tools, and to validate the machine learning coding procedure.

The following three subsections, Dinosaurs names, Shakespeare, and Speech recognition, describe those tests.
\input{content/4.1.1-dino.tex}
\subsection{Shakespeare}

The second test, located in the folder ``coursera-tests/shakespeare'',  was based on another exercise of the same course. In this exercise the model was already trained but there were references to the original source code, 
\todo{reference}, made using Keras. Since neural network training involves some degree of randomization, the previous experiment approach of comparing outputs using unit tests was not applied in this case. The focus of this test was instead to achieve similar results, checking if the network could be trained and, more important, if any problems would arise.

The model of the network uses two LSTM layers with dropout, followed by a fully-connected layer.

The execution of the algorithms was straightforward and provided similar results to the original trained model. It was interesting to notice the usage of variance scaling initializers and dropout on the model, which seemed to give better results compared to results obtained without them.

\subsection{Speech recognition}
The third test, located in the folder ``speech-experiments'', is actually an experiment. Many of the works available using this type of network are focused at speech or text recognition. Therefore speech recognition was chosen as an experiment whose objective was to discover what are the challenges that the use of a LSTM neural network imposes. Using similar works on the speech recognition field, it was expected that the occurrence of eventual problems could be solved using these works as reference or inspiration.

The input of a speech recognition task, which is some kind of wave sound representation, has a very different size when compared to the output, which is text. CTC, as described in section \ref{sec:ctc},
%%%%%
\todo{CTC will be detailed?}
is a convenient solution to perform end to end speech to text conversion because it provides a way to perform back-propagation in this scenario where an unknown number of several consecutive input units are related to to a single output unit.

But, while Keras already has a build-in CTC cost function, called ctc\_batch\_cost, its usage is not as simple as other cost functions of the same framework. These functions only require the predicted and the correct labels, while this CTC cost function also requires the input size and the predicted label size. 

An example included in Kera's own repository
%%%%%
\todo{include reference to imageocr.py}
suggests as a solution to include the loss computation as an extra layer in the network. This layer would receive, besides the previous layer input, the required sizes as two extra inputs, arguing that Keras has no support for a loss function with extra parameters coming from the network. 
While this certainly works, it is unusual. The alternative approach used in this dissertation was to encode the required information as two extra columns on the matrix that stores the correct labels and implement the CTC algorithm as a loss function instead of a layer.
 
Algorithm \ref{alg:ctcloss} shows the code used to split the y\_pred input matrix into three inputs required by Kera's ctc\_batch\_cost function. The result can then be passed to the model.fit() function.

\noindent
\begin{algorithm}
\begin{lstlisting}[frame=single, numbers=left]
def ctc_loss(y_shape):
  def f(y_true, y_pred):
    y_true = tf.reshape(y_true, y_shape)
    k_inputs = y_pred
    k_input_lens = y_true[:,0:1]
    k_label_lens = y_true[:,1:2]
    k_labels = y_true[:,2:]
    cost = K.ctc_batch_cost(k_labels, k_inputs,
        k_input_lens,k_label_lens)
    return cost
  return f
\end{lstlisting}
\caption{\label{alg:ctcloss}ctc\_loss}
\end{algorithm}

The code that prepares the labels matrix, including the sizes as the two first columns, is presented in algorithm \ref{alg:toctcformat}.

\noindent
\begin{algorithm}

\begin{lstlisting}[frame=single, numbers=left]
def to_ctc_format(xs,ys, max_ty=None):
  max_tx = np.max([len(i) for i in xs])
  if max_ty == None:
    max_ty = np.max([len(i) for i in ys]) + 3
  assert max_ty >= np.max([len(i) for i in ys]) + 3
  xarr = np.zeros((len(xs), max_tx, xs[0].shape[1]))
  yarr = np.zeros((len(ys), max_ty))
  for i, x in enumerate(xs):
    xarr[i,:len(x)] = x
  for i, y in enumerate(ys):
    yarr[i,:len(y)+2] = [len(x), len(y), *y]
  return xarr, yarr
\end{lstlisting}
\caption{\label{alg:toctcformat}to\_ctc\_format}
\end{algorithm}

\subsubsection{Objectives}
The main objective of this set of experiments was to 
learn how different algorithms, parameters, and model choices could affect the training time of a neural network model that translates a sound to the correspondent text. The resulting model was not required to generalize well and was not intended to be used in real life applications.

This focus on speed of learning and disregard for generalization simplifies the dataset handling, as it does not avoid overfitting and allows the use of the same dataset for both training and validation. The assumption used is that the training time of a model with the additional requirement of good generalization would not be fastest than the training time of a model without this requirement. Thus, if a model is not able to be overfitted in a acceptable amount of time, it also would not be a good choice for a more robust training session.

This approach makes experimentation fast in the current scenario, where the goal is to explore the technology, to later apply it in a different problem. But this opens the possibility of also using this approach in speech recognition, as it is a fast way to discard models that do not satisfy an acceptable training time performance. Then, the group of remaining experiments could pass to the next stage, where a proper dataset preparation procedure would be used.

In the current set of experiments, the time to reach maximum accuracy was measured. Given that some Portuguese syllables have indistinguishable sounds, like ``ci'' and ``si'', 100\% accuracy would be unreachable and the target value was lowered.

\subsubsection{Datasets}
%%%%%
\todo{references for gTTS and espeak}
Three dataset sources were hypothetically considered to these experiments, recorded audio files of people speaking, Google's Text To Speech (gTTS) API, and espeak software. Espeak was the selected source, as it was the fastest way to generate audio data corresponding to each syllable. Google's gTTS sounds are more realistic and less ``robotic'', but they are generated remotely and have some restrictions on frequency of requests. Also, there may exist licence restrictions on its use, but this was not checked. Recording people speaking syllables would generate a dataset more adequate for generalization, but also would be the most time consuming option.

Three datasets were generated using espeak for syllables in Portuguese. The first contains only the five vowels, 'a', 'e', 'i', 'o', 'u'. The second contains 140 syllables, created using a vowel optionally preceded by one of 27 consonant prefixes, 'b', 'br','c', 'cr','d', 'dr','f', 'fr','g', 'gr','j','l', 'lh','m','n', 'nh','p','pr','qu','r','s','t', 'tr','v','vr','x', and 'z'.
The third adds an optional suffix to the previous syllables, 's', 'r', 'l', and 'm', totalizing 700 syllables.

It is important to notice that the direct use of the written text as a label to each sound, without the use of a phonetic alphabet, was possible because the Portuguese language has a good correspondence between the sound and written representation of each syllable. In other languages, like English, the same approach would not be feasible.

Each generated sound was preprocessed by a Fast Fourier Transform (FFT) algorithm obtained from XXXX,
%%%%%
\todo{reference to ba-dls-deepspeech}
to convert an audio file containing amplitude vs time data to a representation that express frequency vs amplitude vs time. This facilitates the recognition of features by the neural network, as it is expected that the frequency of each sound will have an important role at those features.

\subsubsection{Model}

In the first attempts, the models had only one LSTM layer, usually with 128 units, using dropout of 0.5 and a fully connected layer of 27 outputs. But the results with this configuration, and some variations of it, were unsatisfactory.

The first problem was that the loss would stagnate after some time, resulting in poor accuracy. After the removal of dropout, this behavior was corrected. It is possible that smaller values of dropout could also give the same result, but this alternative was not tested.

But even with a decreasing loss, the network training still was a slow process. It was taking xxxxx 
%%%%%
\todo{rollback and measure}
to learn to recognize only 5 audios of 5 vowels.
Without a similar working model to compare, it was difficult to judge if that training time was normal or not. Some alternative models were tried. While the increase in number of LSTM layers and variations on the quantity of units did not change the results enough, the inclusion of three 1D convolutional layers before the LSTM layer gave dramatic changes in training time.

\subsubsection{Experiment procedure}
Each experiment was separated in a folder, whose name starts with ``v'', ``cv'' or ``cvs'', that indicates the dataset used, and a number to individualize it. On each folder, a Makefile was included to run the experiment.

%%%%%
\todo[inline]{reference to Adam}
All the experiments use the Adam algorithm to perform backpropagation.

The experiments did not take advantage of GPU acceleration and were  conducted on a single computer with 256GB of RAM and with 2 Intel\textregistered Xeon\textregistered E5-2630 v2 processors, with 6 cores each, with 2 hyper-threads per core, or 24 hyper-threads in total. 
The distance measure used in the experiments considered only exact matches as correct answers. As a consequence, the resulting accuracy value only increased above zero after the loss was very low. For a more robust speech recognition application, a less strict distance measure could be a better option.

\subsubsection{v-001}

Experiment v-001 uses a model with three convolutional layers of 16, 8 and 4 units each, followed by a LSTM layer of 64 units and a fully connected layer of 27 output units, with softmax activation.

The code is configured to run for 1000 epochs, with only 5 samples each, one for each vowel, taking 1m08s to run in the hardware used.

\noindent
\begin{algorithm}
\begin{lstlisting}[language=Python, frame=single, numbers=left]
last = l0 = Input(shape=(None,221))
last = Conv1D(16, (3,), padding="same", activation="relu")(last)
last = Conv1D(8, (3,), padding="same", activation="relu")(last)
last = Conv1D(4, (3,), padding="same", activation="relu")(last)
last = LSTM(64, return_sequences=True)(last)
last = Dense(27)(last)
last = Activation('softmax')(last)

model = tf.keras.Model([l0], last)
\end{lstlisting}
\caption{\label{alg:v001}Experiment v-001}
\end{algorithm}

\subsubsection{v-002}
Experiment v-002 uses the same model structure of v-001, but uses an abstract class to handle common tasks and uses callbacks to save checkpoints and to decide when to stop training.

It achieves 1.00 accuracy in 1331 epochs, taking 2m14s.

\subsubsection{v-003}

Experiment v-003 adds more convolutional layers to the previous model.  It also adds a TensorBoard callback, which saves more details of the training process in a format that can be read by TensorBoard and used to perform analysis and create graphics.

It achieved 1.00 accuracy in 601 epochs, taking 0m44s.

\noindent
\begin{algorithm}
\begin{lstlisting}[language=Python, frame=single, numbers=left]
last = l0 = Input(shape=(None,221))
last = Conv1D(16, (3,), padding="same", activation="relu")(last)
last = Conv1D(8, (3,), padding="same", activation="relu")(last)
last = Conv1D(8, (3,), padding="same", activation="relu")(last)
last = Conv1D(8, (3,), padding="same", activation="relu")(last)
last = Conv1D(8, (3,), padding="same", activation="relu")(last)
last = Conv1D(8, (3,), padding="same", activation="relu")(last)
last = Conv1D(4, (3,), padding="same", activation="relu")(last)
last = LSTM(64, return_sequences=True)(last)
last = Dense(27)(last)
last = Activation('softmax')(last)

model = tf.keras.Model([l0], last)
\end{lstlisting}
\caption{\label{alg:v003}Experiment v-003}
\end{algorithm}

\subsubsection{v-004}
Experiment v-004 adds a maxpool layer after each convolutional layer. The ``data\_format=`channels\_first`'' option specifies the dimension that should be reduced, which should be the data of a single time step. With the default option the MaxPool1D function would reduce the number of time steps.

It achieved 1.00 accuracy in 501 epochs, taking 0m40s.

\noindent
\begin{algorithm}
\begin{lstlisting}[language=Python, frame=single, numbers=left]
maxpool = MaxPool1D(pool_size=2,
                    strides=1,
                    data_format='channels_first')
last = l0 = Input(shape=(None,221))
last = Conv1D(16, (3,), padding="same", activation="relu")(last)
last = maxpool(last)
last = Conv1D(8, (3,), padding="same", activation="relu")(last)
last = maxpool(last)
last = Conv1D(8, (3,), padding="same", activation="relu")(last)
last = maxpool(last)
last = Conv1D(8, (3,), padding="same", activation="relu")(last)
last = maxpool(last)
last = Conv1D(8, (3,), padding="same", activation="relu")(last)
last = maxpool(last)
last = Conv1D(8, (3,), padding="same", activation="relu")(last)
last = maxpool(last)
last = Conv1D(4, (3,), padding="same", activation="relu")(last)
last = maxpool(last)
last = LSTM(64, return_sequences=True)(last)
last = Dense(27)(last)
last = Activation('softmax')(last)

model = tf.keras.Model([l0], last)
\end{lstlisting}
\caption{\label{alg:v004}Experiment v-004}
\end{algorithm}

\subsubsection{cv-001}
Experiment cv-001 is based on experiment v-001, but applied the model on a dataset of syllables consisting of a consonant followed by a vowel. It uses a stop condition of accuracy $>=$ 0.6.

The experiment took 3m52s to reach an accuracy of 0.6, but the loss presented stagnation after that.

However, it was observed that the reuse of the model already trained with only the vowels improved the training time. In the current experiment the same model trained in experiment v-001 was used as a starting point in experiment cv-001.

\subsubsection{cv-002}
Experiment cv-002 also uses a source code based on experiment v-002, just changing the dataset and batch size.

A key difference between experiment cv-001 and cv-002 is that in the first, the ``fit'' function was called many times, one for each epoch, while in the later, the ``fit'' function was called only once, specifying the maximum number of epochs that should be run. The disadvantage of calling the ``fit'' function many times is that the internal state of the optimizer algorithm is reset at each interaction, thus causing the observed stagnation of the loss value.

Using a model previously trained with vowels, it achieved 0.90 accuracy on 140 syllables in 4301 epochs , taking 8m6s.

\subsubsection{cv-003}

Experiment cv-003 also uses a source code based on  experiment v-002, but the inclusion of items on the training sample was done gradually. Given that in cv-001 it was observed that training the vowels first improved training time, in experiment cv-003 the initial training dataset included only the vowels first. Then, a new item was added to the training set each time the loss was less than a certain threshold. This was implemented in Keras using a ``fit\_generator'' function and callbacks.

It was noticed that the value of the loss threshold to trigger the increase of training data has an important impact on the overall training time. But none of the values tried, 1.0, 0.5, and 0.3, generated satisfactory results, as in all three cases the training was interrupted after one hour, as it presented a clear disadvantage when compared to the previous experiment, that finished in less than 10 minutes.

\subsubsection{cvs-001}

Experiment cvs-001 uses a source code based on experiment cv-002, which is based on v-002. The difference was the dataset, that also included the suffixes ``s'', ``r'', ``l'', and ``m''.

Using the model previously trained in experiment cv-002, it achieved 0.90 accuracy on 700 syllables in 501 epochs, taking 5m8s.

\subsubsection{cvs-002}
Experiment cvs-002 is identical to cvs-001, but uses the model trained by v-002 instead of the one trained by cv-002.

It achieved 0.90 accuracy on 700 syllables in 1601 epochs, taking 16m21s.


\chapter{\label{chap:experiments}Experiments}
\chapter{\label{chap:methods}Experiments}

\section{Filetype identification}
The objective of the experiments described in this section was to investigate which neural network models would be more adequate to the filetype identification task and to be used in subsequent research stages.
\subsection{Datasets}







train
dev (eyeball, blackbox)
test

training set from different source

dev and test sets from same source

Datasets:
- govdoc1: 1000 subsets. use minimal subsets as possible during training.
Start with subset only two subsets dev and test. Add other subsets do dev later if needed.

- individual files gathered from multiple sources : train, dev, test
on demand, as govdoc begins to fail on dev set




\subsection{Exp01 - 3 files, by sectors}
3 files
identification of sectors

    \section{Objective}
    The objective of the experiments described in this section 
was to compare the training time of different neural networks, given an target accuracy level, at the filetype identification task.
    \section{Methodology}
    \input{content/5.3.2-methodology.tex}
    \section{Datasets}
    \input{content/5.3.3-datasets.tex}
    \section{Environment}
    The experiments did not take advantage of GPU acceleration and were  conducted on a single computer with 256GB of RAM and with 2 Intel\textregistered Xeon\textregistered E5-2630 v2 processors, with 6 cores each, with 2 hyper-threads per core, or 24 hyper-threads in total. 

The source code for the experiments is available at \url{http://github.com/atilaromero/ML}.


    \section{Results}
    \input{content/5.3.5-results.tex}
        \subsection{First block as input}
        \input{content/5.3.5.1-firstblock.tex}
        \subsection{Random block as input}
        The next set of experiments uses a random block (512 bytes chunks) from each file, instead of just the first one. This is a harder classification task because, while in the first block is reasonable to find patterns in specific positions in relation to the beginning of the block, this correspondence is not normally preserved in the remaining blocks, as the pattern may start anywhere in the block. The second factor is that in files with low compression rates, as image files, the beginning of the file normally presents more recognizable patterns than the middle.

Each epoch was configured to draw 1000 samples from the training dataset. Validation was performed using 1000 samples from the development dataset. Each sample had only 512 bytes.

Comparing the three network structures used in the previous set of experiments, the ``CL'' network had the best results.
%12 D
The simple feedforward network, ``D'', that performed well classifying the first block was unable to achieve similar results when classifying a random block.
To check if the accuracy would improve with more training, the network was later trained for 600 epochs, but it only reached an accuracy of 77.5\% on the validation dataset, taking 62 minutes.

%LD
The network using only a LSTM layer followed by a fully connected layer, ``LD'', was the slowest to train, and achieved a low accuracy when compared to the other two networks.

%CL
The network ``CL'', that used a convolutional layer to divide the input block in 16 smaller blocks of 32 bytes and used the LSTM layer to process the results gave best results of the three.

\begin{table}[!ht]
    \centering
    \caption{Filetype identification experiments with random block}
    \label{tab:carvingrandomblock}
\begin{tabular}{r|r|r|r|r|r|r}
\hline
Name & Parameters & Blocks & Epochs & Time    & Training          & Validation          \\       
     &            &        &        &         &          accuracy &            accuracy \\ \hline\hline

D  & 393219 & all & 150 & 7m45s  & 0.758 & 0.684 \\ \hline
LD & 37091  & all & 13  & 10m10s & 0.462 & 0.468 \\ \hline
CL & 24663  & all & 150 & 8m38s  & 0.813 & 0.783 \\ \hline
\end{tabular}
\end{table}

\begin{figure}[htb!]
\centering\includegraphics[width=0.65\textwidth]{content/random-block.png}
\caption{\label{fig:randomblock}Experiments with random blocks}%
\end{figure}
\todo[inline]{legenda}


        \subsection{CL variations}
        \input{content/5.3.5.3-cl-variations.tex}
        \subsection{No LSTM}
        \input{content/5.3.5.4-nolstm.tex}
        \subsection{Two convolutional layers}
        To test if an increase on the deepness of the network would improve its results, four networks using two convolutional layers were experimented. 

At the ``CCL'' network, the first convolutional layer uses 8 units for the input window, and a stride of the same size, thus dividing the 512 bytes of the input block in 64 matrices of 8x256 that are then converted in 64 vectors of 128 units. The second layer divides the 64x128 input in 8 matrices of 8x128, producing 8 vectors of 64 units. The LSTM layer uses each of these 8 vectors as time steps, producing a single vector of 3 units, one for each class.

The ``CCLL'' network adds a LSTM layer of 64 output units between the convolutional layers and the final LSTM layer.

The ``CMCML'' network is similar to ``CCL'', but uses max pooling of size 2 at each convolutional layer.

The ``CMCMLL'' network combines the two additions, using max pooling and using two LSTM layers.

The results are shown in table \ref{tab:carving2convs}. The ``CL'' network is included for comparison. The ``CCL'' results were worst than ``CL'' ones, while ``CCLL'' were better. But the best results of all experiments described so far were those
These three networks produced similar results, but they are not as good as the ``CL'' network, as can be seen in figure
\ref{fig:nolstm}.

\begin{table}[!ht]
    \centering
    \caption{Two convolutional layers}
    \label{tab:carving2convs}
\begin{tabular}{r|r|r|r|r|r|r}
\hline
Name & Parameters & Blocks & Epochs & Time & Training          & Validation          \\       
     &            &        &        &         &          accuracy &            accuracy \\ \hline\hline

CL & 24663  & all & 150 & 8m38s  & 0.813 & 0.783 \\ \hline
CCL    & 328688 & all & 113 & 10m04s & 0.762 & 0.766 \\ \hline
CCLL   & 361712 & all & 98  & 10m02s & 0.853 & 0.823 \\ \hline
CMCML  & 295536 & all & 84  & 7m42s  & 0.902 & 0.858 \\ \hline
CMCMLL & 320752 & all & 94  & 10m01s & 0.887 & 0.871 \\ \hline
\end{tabular}
\end{table}

\begin{figure}[htb!]
\centering\includegraphics[width=0.65\textwidth]{content/twoconvs.png}
\caption{\label{fig:twoconvs}Two convolutional layers}%
\end{figure}
\todo[inline]{legenda}

    \section{Discussion}
    
Table \ref{tab:carvinglayers} specify the number of output units in each layer of the networks used in the experiments. For the networks with two convolutional layers, those layers are connected together before the LSTM layer. For the convolutional layers, the input number indicates the size of the receptive field. 
\begin{table}[!ht]
    \centering
    \caption{Experiments layers}
    \label{tab:carvinglayers}
\begin{tabular}{r|r|r|r|r|r|r}

       & \multicolumn{4}{|c|}{}                         &        & Fully           \\     
       & \multicolumn{4}{|c|}{Convolutional}            & LSTM   &       connected \\ \hline
Name   & Input         & Stride & Output & Pooling size & Output & Output          \\ \hline\hline

D      &               &        &        &              &        & 3               \\ \hline
LD     &               &        &        &              & 32     & 3               \\ \hline
CL     & 32            & 32     & 3      &              & 3      &                 \\ \hline
\hline
CLL    & 32            & 32     & 32     &              & 64     &                 \\       
       &               &        &        &              & 3      &                 \\ \hline
CML    & 32            & 32     & 32     & 2            & 3      &                 \\ \hline
CLD    & 16            & 16     & 256    &              & 128    & 3               \\ \hline
\hline
CM     & 32            & 1      & 3      & 481          &        &                 \\ \hline
CCM    & 32            & 1      & 3      &              &        &                 \\       
       & 2             & 2      & 3      & 240          &        &                 \\ \hline
CD     & 64            & 8      & 64     &              &        & 3               \\ \hline
\hline
CCL    & 8             & 8      & 128    &              & 3      &                 \\       
       & 8             & 8      & 64     &              &        &                 \\ \hline
CCLL   & 8             & 8      & 128    &              & 64     &                 \\       
       & 8             & 8      & 64     &              & 3      &                 \\ \hline

CMCML  & 8             & 8      & 128    & 2            & 3      &                 \\       
       & 8             & 8      & 64     & 2            &        &                 \\ \hline
CMCMLL & 8             & 8      & 128    & 2            & 64     &                 \\       
       & 8             & 8      & 64     & 2            & 3      &                 \\ \hline

\end{tabular}
\end{table}
% \chapter{\label{chap:proposedsolution}Proposed solution}

% \todo[inline]{hypothesis on 2.1: amount of work}
% \todo[inline]{methodology (should be a chapter?)}

% \chapter{\label{chap:validation}Validation}

\chapter{\label{chap:conclusion}Conclusion}
\chapter{\label{chap:conclusion}Conclusion}
\section{Future work}
    \section{Future work}
    \input{content/8.1-futurework.tex}