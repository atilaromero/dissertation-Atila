Normally, computer users do not need to deal with hard disk sectors directly and have contact only with the already mounted filesystem, which presents directories and files for them. But to present such a view, the operating system has to interpret the raw media data, which is simply a stream of data blocks.

The first blocks of a drive usually contain a partition table, indicating ranges of blocks belonging to each partition. Inside a partition, the operating system expects to find a filesystem. The filesystem stores metadata about each file or directory and keeps an index indicating the position of each file on disk. This way, when a user access a file, the operating system uses the filesystem to find where the file is stored on the disk, accessing those areas directly and returning its content to the user, who sees the returned data as the file content. 

When a file is deleted, the corresponding index entry is erased, while the actual content of the file may be left untouched to avoid disk access. In this circumstance, a data carving procedure may successfully retrieve the file, even if the filesystem cannot. One common data carving approach that does not deal with fragmentation consists of searching for headers and footers. To retrieve the file, a software using this approach would sequentially read each drive sector, find a known header and save the following sectors until a footer is found or a size limit is reached. 
