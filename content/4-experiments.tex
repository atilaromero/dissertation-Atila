% \levelC{Dataset}
This study uses the Govdocs1 dataset \todo{citation}, which was fully downloaded and its files were grouped by extension. This dataset has files with 63 different extensions. The 33 extensions with less than 200 files were discarded. From the remaining 30 extensions, listed in table \ref{tab:govdocs1}, 200 files of each were randomly selected, 100 to use in the training dataset and 100 to use in the validation dataset.

\input{content/tables/4.0.1-govdocs.tex}

% \levelC{Hardware}
The experiments did not take advantage of GPU acceleration and were  conducted on a single computer with 256GB of RAM and with 2 Intel\textregistered Xeon\textregistered E5-2630 v2 processors, with 6 cores each, with 2 hyper-threads per core, or 24 hyper-threads in total. 


% \levelC{Software}
\todo[inline]{python, keras, tensorflow, linux, jupyter notebook}

% repository
The source code for the experiments is available at \url{http://github.com/atilaromero/ML}.
\todo[inline]{create a cleaner repository just for the paper}

Section \ref{sec:evalmodels} evaluates some alternative models in the file fragment classification task. The expectation was to identify the most promising models to continue improvement. Instead, an apparent limit on how far those models could be improved was found. 

Section \ref{sec:numberofclasses} \todo{XXXX}

Section \ref{sec:exprandom} \todo{XXXX}



\levelB{Research on accuracy of some models}
% Objective:
In this experiment, different neural networks are trained and evaluated
at the filetype identification task. Their accuracy is then compared,
to identify which models should be considered or disregarded in the remaining phases.

% Inputs:
The input features of the network for each instance is a 512x8 matrix representing only one block of 512 bytes of a random file of the dataset. Each of the 512 bytes uses a custom encoding where each of the 8 bits of a byte is represented as -1 or 1, depending whether the bit is 0 or 1. During initial tests, this encoding was compared to three other encodings: one-hot encoding, 8 bits represented as 0 or 1\todo{include citation}, and 8 bits represented as [0,1] and [1,0] \cite{hiester_file_2018}. More research should be done in the future to determine the best of the four, but initial results suggest they have similar impact on the model accuracy. The one-hot encoding has the disadvantage of increasing the input matrix size by a factor of 32.

% outputs:
The output of the network for a given instance is a vector with a size equal to the number of classes, subjected to a softmax function, which applies the exponential function on the vector and then normalizes\todo{check if this is normalization} it. Each value will represent the predicted probability that the instance belongs to each of the 
XXXXX
three classes, PDF, JPG or PNG.


% Dataset:


%Constraints
The models where trained with a time constraint of ten minutes, and using only a small training dataset,

Models:


Dataset preparation:

%Sampling


Setup:
--- software, hardware, algorithm
--- vm, os, application, memory, HD

Results:

Limitations and threats to validity:


Minimize the training time is important when the task of create new models to new filetypes is delegated to the forensic examiners community, since they may not have expensive hardware or time to train a more demanding neural network.

For similar reasons, networks that require too much tuning to perform well are undesirable, since it would require a higher knowledge and work from the forensic examiner.


%metricas
To identify networks with such requirements, each experiment use two stop conditions, ten minutes, or an accuracy of 90\% in the training set, whichever occurs first. These limits were established during the experiments.

\todo[inline]{justificar critério de parada: estabilização/estagnação; quando vai aumentando o tempo, não existia mais modificações, ou seja 10 minutos chegavam para a estabilidade do modelo}

%Since the rate of change of accuracy may present oscillations during training, a visual comparison of accuracy versus time curves is important to confirm which network has better training times.

All the experiments use the Adam \cite{kingma_adam:_2014}
optimization algorithm to guide backpropagation, which was selected because it performed well in the preliminary results without requiring tuning the learning rate.

%modelos
The networks considered used different combinations of convolutional, max pooling, LSTM, and fully connected layers.

The experiment names are based on the types of layers that compose each architecture and their purpose is to differentiate one experiment of another. Thus, they are not intended to be used outside of this work.






Dissemination:
-- dataset
-- authorship
-- repository
-- id and citation
-- data management plan
-- diary
-- experimental issues


\levelB{Research on influence of number of classes}


\levelB{Experiment on random data detection}
\label{sec:exprandom}

In the previous sections, it was observed that the higher the number of classes being considered during the creation of a file fragment classification model, the higher is the error rate of this model. In addition, the comparison of pairs of classes indicated that high entropy data structures may be responsible by some part of those errors.

Prior to conducting experiments to find the major cause of error, a list of conceivable error sources (E1 to E4) was elaborated:
\begin{enumerate}[itemindent=\parindent,label=\textbf{E\arabic*.}]
    \item For some data structure, the model cannot distinguish it from random data. This can happen if the pattern in the data is too complex, beyond the capabilities of the model. It may be the case that in practice no model can perform this distinction or, instead, this may be a limitation of this particular model only. This situation may occur in files that use compression or cryptography, or more generally, any filetype with high entropy.

    \item Different filetypes using the same data structure. It is common to a given filetype to employ different types of data structures. If two or more filetypes make use of the same data structure, the existence of that structure will then not be sufficient to differentiate those filetypes, as it may belong to anyone of them.
    The reverse, a filetype that uses multiple kinds of data structures, does not constitute a problem as it simply results in extra work during the model training.

    \item Same filetype with multiple extensions. If the same filetype appears in the dataset with multiple extensions, ``JPG'' and ``JPEG'' for example, the model will not be able to predict which label is used in the validation dataset for a given instance, since this distinction exists in the labelling but not in the content.

    \item Files that contain other files. Some filetypes need to embed other files. If the inner file is categorized, even if the model finds an unmistakable pattern, it will not match the label, which will be the extension of the outer file. A ``PDF'' file, for example, can embed a ``JPG'' file. If the model predict ``JPG'' as the class of a portion of this file, it will no match the instance label on the dataset, which would be ``PDF''.
\end{enumerate}

The three last error sources listed above are similar in nature, as the last two may be viewed as special cases of E2. This error source comes from the practice of using the extension of the file as the class to each of its parts. An analogy with speech recognition would be to label each syllable of a spoken word using the word as its label and then try to predict the whole word using only a syllable.

Unfortunately, for the file fragmentation task the potential labels for smaller parts may be less obvious than it is for speech recognition. The best case scenario would be if the neural network itself could choose the labelling. But evaluate those predicted labels using labels of its own choosing would introduce bias, as it could prefer to create only easy labels. Saying that all filetypes are composed of ``data'' is correct, but it is unhelpful.



\levelC{Objective}


\levelC{Dataset}
\levelC{Sampling}
\levelC{Inputs}
\levelC{Outputs}
\levelC{Models}
\levelC{Results}
\levelC{Limitations and threats to validity}

% The extensions ``text'' and ``unk'' were also discarded because they do not correspond to a file type, and the files that use them belong to assorted types. 


% estudo quantitativo exploratório

% 0 - hipotese:
%     fontes de erro:
%     - padrões irreconhecíveis em dados de alta entropia
%     - padrões reconhecíceis, mas modelo falha
%     - multíplos arquivos usando mesmas estruturas de dados
%     - arquivos com múltiplas extensões
%     - arquivos que contém outros arquivos
%     - extensões erradas

% 1 - naive comparison of models using 31/35 classes
%     drop in accuracy with increase o number of classes
    
% 2 - file command

% 3 - comparison to random




% ========

%  - melhor modelo - tipos de camadas
%  -- dataset juntado
%  --- 3 classes
 
%  - comparacao com outros trabalhos
%  -- govdocs1
%  --- 4, 20 classes
 
%  - numero de classes
 
%  - comparison to random
 
%  - text e unk
 
%  - one-hot, 8bits01, 8bits-11, 16 bits

%  - softmax, sigmoid, categorical crossentropy, Binary crossentropy, mse
 
%  - mask
 