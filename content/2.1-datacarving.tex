\section{\label{sec:datacarving}Data carving}
% \subsection{Description}
%from pep 1, paragrafo 2
Data carving is a forensic process that attempts to recover files without previous information of where the file starts or ends \cite{garfinkel_carving_2007}.
To accomplish this, a software has to analyze a source of raw data, searching for patterns indicating a known file type and making attempts to locate and reconstruct each of its constituent parts.

%from pep 1, paragrafo 2b
That process commonly disregards the filesystem \cite{veenman_statistical_2007}, being able to recover deleted files from unallocated areas, but faces the problem of fragmentation \cite{veenman_statistical_2007}  \cite{pal_evolution_2009}: in many cases, files are not written sequentially on disk and deleted files may have missing parts.

This procedure is frequently used in forensic environments, but it may also be beneficial in other areas, such as reverse engineering, network traffic analysis, and data mining.

This observation is related to the fact that many types of data sources contains embedded files. Therefore, they may be used as input to a data carving process. This includes network traffic, memory dumps, hard drive images, and files containing other files.

To show how a data carving process work, a hard drive with a deleted video file is here used as an example.
Normally, the average user of a computer does not need to deal with hard disk sectors directly and has contact only with the already mounted filesystem, which presents directories and files for the user. But in order to present such view, the operating system has to interpret the raw hard drive data, which is simply a stream of data blocks. The first blocks of the drive contain a partition table, indicating ranges of blocks belonging to each of the drive partitions. Inside a partition, the operating system expects to find a filesystem. The filesystem stores metadata about each file or directory and keeps an index indicating the position of each file on disk. This way, when a user opens a file, the operating system uses the filesystem to find where in the disk the file is stored, accessing those areas directly and returning its content to the user, who sees the returned data as the file content. 
 
If video file in the example had not been deleted, the filesystem could use the index to find the video location on disk. But after deletion, the corresponding index entry is erased, while the actual content of the file may be left untouched to avoid disk access. In this circumstance, a data carving procedure may successfully retrieve the file, even if the filesystem cannot. One common data carving approach that does not deal with fragmentation consists in searching for headers and footers patterns. To retrieve the hypothetical video file in the example, a software using this approach would sequentially read each drive sector, find a known header of a video file and save the following sectors until a footer is found or a size limit is reached. 
 
\subsection{Data carving evolution}
Ali et al. \cite{ali_review_2018} divide the data carving process in three steps:
    identification, which classifies the file type of individual chunks of data; 
    validation, which includes a list of requirements of a file that are needed for its recovery to be considered successful; and
    reassembling, which attempts to reconstruct the original file.

Nadeem \cite{nadeem_ashraf_forensic_2013} groups the carving techniques in three generations, each extending the previous one.
The first generation is a header-footer based carving. It uses file signatures like magic-bytes, headers, and footers to identify the beginning and end of a file.
The second generation is structure based carving, also called ``semantic carving'' or ``deep carving''. It reduces the number of false positives by using file structure knowledge to perform validation.
The third generation advances reassembling with methods to deal with fragmentation. It tries to infer relationships and order between chunks of data based on content and statistical analysis to reassemble the original file.

For file type detection, which could be mapped to the identification step in the  process steps of Ali et al. \cite{ali_review_2018}, Amirami et al. \cite{amirani_new_2008} cite three categories: extension-based identification, magic bytes-based identification, and content-based identification.
In extension-based identification, the content of the file is ignored and only its filename extension is used. Magic bytes-based identification uses signatures, generally a fixed string, usually at the beginning of a file. It is a common strategy that uses header/footer, but not all files adopt it. Content-based identification identifies the file using some statistical modeling of its content.

Beebe et al. \cite{beebe_sceadan:_2013} identify three content-based approaches to classify file and data types, also referring only to the identification step of the Ali et al. \cite{ali_review_2018} data carving process division: semantic parsing, nonsemantic parsing, and machine learning. Semantic parsing relies on the file structure to identify its type. Nonsemantic parsing searches for strings that are commonly found in specific files. Machine learning uses supervised and unsupervised algorithms, like Support Vector Machine (SVM), k-Nearest Neighbors (kNN), and Neural Networks (NN).

\sigla{NN}{Neural Networks}
\sigla{kNN}{k-Nearest Neighbors}
\sigla{SVM}{Support Vector Machine}

A summary of the categorization schemes is depicted on Table \ref{tab:categories}. As Nadeem \cite{nadeem_ashraf_forensic_2013} does not specifically mention machine learning, it is left out of generation classification.

\begin{table*}[!ht]
    \centering
    \caption{Data carving categories}
    \label{tab:categories}
    \begin{tabular}{ l | l | l | c }
      \multicolumn{3}{l|}{Steps}                                 & Generation\\
      \hline\hline
      Identification    & extension-based   &                   &   \\
                        \cline{2-3}
                        & magic bytes-based &                   & \multirow{-2}{*}{1\textsuperscript{st}}\\
                        \cline{2-4}
                        & content-based     & semantic          &   \\
                                            \cline{3-3}
                        &                   & non-semantic      & \multirow{-2}{*}{2\textsuperscript{nd}}\\
                                            \cline{3-4}
                        &                   & machine learning  &  --- \\
      \hline
      Validation        &                   &                   & 2\textsuperscript{nd} \\
      \hline
      Reassembling      &                   &                   & 3\textsuperscript{rd}\\
      \hline
    \end{tabular}
\end{table*}




\subsection{Current data carving tools}

%from pep 4, paragrafo 2
Some studies list available data carving tools
\cite{ali_review_2018}
\cite{qiu_new_2014}
\cite{nadeem_ashraf_forensic_2013}
\cite{roux_reconstructing_2008}, 
but the tool listing of Ali et al. \cite{ali_review_2018} was found to be the most comprehensive. Among the listed tools, only Foremost \cite{kendall_foremost_2019}, Scalpel \cite{richard_iii_scalpel:_2005}, and PhotoRec \cite{grenier_photorec_2019} support a wide range of file formats. Photorec supports more than 300 file types. But these three tools rely mainly on header/footer signature identification.

%from pep 4, paragrafo 1
The available data carving tools generally do not take advantage of the latest techniques that research on the field offers, often still relying on header/footer identification and providing limited reassembling capabilities.
%from pep 2, paragrafo 11
According to Ali et al. \cite{ali_review_2018}, artificial intelligence techniques are found to be not fully utilized in this field.

%from pep 4, paragrafo 3
The amount of work required to support the vast amount of file types in existence is here chosen as a hypothesis for the reason for this discrepancy. Most of the attention paid to the results of data carving research is focused on increasing some statistical measurement of success, like accuracy. While these advances are undoubtedly important, they may not be the kind of research needed to transform these findings into practical tools. If this is the case, then the forensic community would benefit from researches that could make the task of supporting the carving of a new file type easier. Machine learning techniques have the potential to achieve that goal because it can replace the step of manually encoding a structure parser by automatically recognizing patterns in large amounts of data.



\subsection{Neural networks research in data carving}

Amirami et al.  \cite{amirani_new_2008} appear to be the first 
to provide a viable alternative to classical data carving tools using a neural network approach. Two previous works were found using neural networks with data carving related goals, by Dunhan et al. \cite{dunham_classifying_2005} and Harris \cite{harris_using_2007}, but the first worked with encrypted files only and the second did not achieve good results.

% In 2008, 
Amirami et al.  \cite{amirani_new_2008} used Principal Component Analysis (PCA) as input for a 5 layer feed-forward auto-associative unsupervised neural network to do feature extraction and a 3 layer Multi Layer Perceptron (MLP) to perform classification. They used a similar approach in 2013 \cite{amirani_feature-based_2013}.
\sigla{PCA}{Principal Component Analysis}
\sigla{MLP}{Multi Layer Perceptron}

Other works were found applying neural networks to perform data carving, also using some form of dimensionality reduction as PCA. Ahmed et al. \cite{ahmed_content-based_2010}\cite{ahmed_fast_2011} used byte frequency, 
Penrose et al. \cite{penrose_approaches_2013} used compression rate,
and Maslim et al. \cite{maslim_distributed_2014} used PCA, as did Amirami et al.  \cite{amirani_new_2008}.

Xu and Dong \cite{xu_reassembling_2009} used a neural network as a cluster reassembling technique for JPEG image fragments.

Hiester \cite{hiester_file_2018} apparently was the first to not resort to dimensionality reduction and also the first to utilize a LSTM network to perform file fragment classification. He compared results using three types of neural networks: feedforward, convolutional, and long short-term memory. The goal was to classify the data type of individual sectors (512 bytes), considering four file types: CSV, XML, JPG and GIF.

% Table \ref{tab:datacarvingstudies} summarizes the  machine learning techniques used in each data carving study.
% \begin{table}[!ht]
\caption{Data carving studies using machine learning}
\label{tab:datacarvingstudies}
\begin{tabular}{l|c|c|c|c}

                                & \multicolumn{4}{c}{Technique} \\ \hline
Study                                     & SVM    & kNN    & NN   & ELM   \\ \hline
\hline
Dunham et al. \cite{dunham_classifying_2005}       &        &        & x    &       \\ \hline
Harris \cite{harris_using_2007}             &        &        & x    &       \\ \hline
Amirani et al. \cite{amirani_new_2008}              &        &        & x    &       \\ \hline
Xu and Dong \cite{xu_reassembling_2009}          &        &        & x    &       \\ \hline
Ahmed et al. \cite{ahmed_content-based_2010}      &        &        & x    &       \\ \hline
Axelsson \cite{axelsson_normalised_2010}      &        & x      &      &       \\ \hline
Conti et al. \cite{conti_automated_2010}          &        & x      &      &       \\ \hline
Ahmed et al. \cite{ahmed_fast_2011}               & x      & x      & x    &       \\ \hline
Gopal et al. \cite{gopal_statistical_2011}        & x      & x      &      &       \\ \hline
Luigi and Stefano \cite{luigi_file_2011}               & x      &        &      &       \\ \hline
Sportiello and Zanero \cite{sportiello_file_2011}          & x      &        &      &       \\ \hline
Fitzgerald et al. \cite{fitzgerald_using_2012}         & x      &        &      &       \\ \hline
Sportiello and Zanero \cite{sportiello_context-based_2012} & x      &        &      &       \\ \hline
Amirani et al. \cite{amirani_feature-based_2013}    & x      &        & x    &       \\ \hline
Beebe et al. \cite{beebe_sceadan:_2013}           & x      &        &      &       \\ \hline
Penrose et al. \cite{penrose_approaches_2013}       &        &        & x    &       \\ \hline
Qiu et al. \cite{qiu_new_2014}                  & x      &        &      &       \\ \hline
Maslim et al. \cite{maslim_distributed_2014}       &        &        & x    &       \\ \hline
Zhang et al. \cite{zhang_svm_2016}                & x      &        &      & x      \\ \hline
Ali et al. \cite{ali_classification_2018}       &        &        &      & x     \\ \hline
Hiester \cite{hiester_file_2018}             &        &        & x    &       \\ %\hline
\end{tabular}
\end{table}
