\section{Motivation}
% establishing the context, background and/or importance of the topic
    
% file recovery: motivation
% data carving: sometimes the only solution
%from pep 1, paragrafo 1
In a forensic context, file recovery is a frequent task that can be motivated by several situations, like physical media malfunction, intentional attempt to hide data, and the need to access deleted or older versions of files. When the filesystem no longer provides the physical location of a file on the media, data carving is often the only procedure capable of retrieving its content.


% the problem of data carving development
%from pep 1, paragrafo 4
The patterns searched by data carving software are generally manually coded, taking advantage of fixed byte sequences found on headers and footers. But the amount of different file types combined with the slow process of manually coding each of those patterns makes the development of data carving software a tedious task \cite{mcdaniel_content_2003}.

% ml as a solution
%from pep 1, paragrafo 5
The application of machine learning solutions to this manual task has the potential to make it easier and faster. An initial strategy could be to train a classifier to, given a chunk of data, provide a label indicating a file type. That could be used to recover unfragmented deleted files.
% novo
Then, that same classifier can be applied in small chunks of data to produce input to a second algorithm, responsible to reassemble the fragments of a file.

% ml as a solution
%from pep 1, paragrafo 6
% The recovery of fragmented files through data carving would require some sort of pattern recognition on the identified chunks, in order to reconstruct the correct sequence.
