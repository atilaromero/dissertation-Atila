For future research on file fragment classification, the hypothesis that the remaining 2/3 of the observed errors are caused by similar data structures used by multiple files should be explored.

Also, the search for a procedure to automatically label inner data structures of files may be a promising strategy. The automatic identification of inner data structures is specially interesting because it could help to interpret the data.

For future research on reassembling, an LSTM that uses the file fragment classification scores as input is an option. Another potentially helpful approach to reassembling would be a neural network to predict, for an specific file type, which bytes belongs to the file and which do not can be used to detect the start and the end of the file.

% Having recognized the potential of neural networks in the data carving task, there are still some research paths that should be explored.

% \todo[inline]{qual a dificuldade da tarefa de identificacao do tipo de arquivo, o que os outros metodos atingem de acuracia?}

% \todo[inline]{aumentar número de tipos de arquivos suportados}

% \todo[inline]{Existem aspectos que já avançaram mas não citados no texto, por exemplo, já consegue identificar o início e o fim do arquivo.}

% One of them is about the increase in the range of supported filetypes. In a collaborative approach, should the said community be sharing models, datasets, or both? What are the strengths and weakness of each?

% Another one is about reassembling. After each block has been classified, how to reconstruct the original file in the occurrence of fragmentation?

% \todo[inline]{Quanto a remontagem? Existe uma previsão e qual o dataset tu vais utilizar? Como vai conseguir remontar? Este problema é complicado, mas deve tentar avançar de alguma forma.}

% \todo[inline]{Pretende ainda avançar nos tipos de arquivos e na remontagem. Mesmo que não consiga remontar, pelo menos mostrar o que não conseguiu, pois facilita para quem for feito. —> Verificar a literatura se ninguém está fazendo isto mesmo.}

% The last one is about the recognized structures. Is it possible to use the trained networks to describe the structures being recognized?

% ==========


% The following questions are intended to be answered in future works: 

% \begin{enumerate}[itemindent=\parindent,label=\textbf{Q\arabic*.}]

%     \item Could a neural network based tool support a wider range of file types?
    
%     \item Could a neural network based tool handle fragmentation through reassembling?
    
%     % \item Do the results obtained with usual datasets reflect what happens in real scenarios?

% \item Do neural networks help to interpret internal file structures?

% \end{enumerate}

% \todo[inline]{compare solutions - possible candidates: feedforward, convolutional, LSTM, BLSTM, SVM, kNN, Photorec, Foremost, scalpel}
% \todo[inline]{shuffle data to simulate fragmentation}
% \todo[inline]{removal of portions of files to simulate data corruption}
% \todo[inline]{increase the number of supported file types, investigating the best strategy to scale the solution}
% \todo[inline]{reassembling}
% \todo[inline]{model share}
% \todo[inline]{adaption of visualization techniques of neural networks, attempting to infer file structure.}