\begin{resumo}{computação forense, \textit{data carving}, aprendizado de máquina, redes neurais, Long Short-Term Memory, camadas convolucionais, classificação de fragmentos de arquivo, dados de alta entropia}

Classificação de fragmentos de arquivos visa identificar o tipo original de um arquivo de onde um bloco de dados foi extraído. As técnicas de aprendizado de máquina, e as redes neurais em particular, têm o potencial de aprimorar esta área, porque o suporte a um novo tipo de arquivo usando métodos tradicionais é trabalhoso e não é automático. Embora estudos sobre redes neurais aplicadas à classificação de fragmentos de arquivos tenham mostrado bons resultados, para aplicar essas contribuições em cenários reais é necessário um melhor entendimento de como esses métodos respondem a um aumento no número de tipos de arquivos suportados e quais são as principais fontes de erros desta abordagem.

Este trabalho foca nos desafios da aplicação de redes neurais para classificação de fragmentos de arquivos e está organizado em três partes.
Na primeira, os valores de acurácia de alguns tipos de redes neurais são comparados, classificando fragmentos de arquivo extraídos do \textit{dataset} Govdocs1 \cite{garfinkel_bringing_2009}, usando suas extensões de arquivo como classes.
Os resultados sugerem que os tipos de arquivo que compõem o \textit{dataset} têm um papel mais relevante nos resultados do que os detalhes de arquitetura dos modelos de redes neurais.
Na segunda parte, a influência do número de classes na acurácia é explorada.
A conclusão é que o número de classes no \textit{dataset} é relevante, mas menos importante que os tipos de extensões selecionadas para compô-lo.
Finalmente, um procedimento é criado para testar a hipótese de que parte dos erros de classificação de fragmentos de arquivos pode ser explicada pela incapacidade dos modelos de distinguir dados de alta entropia de dados aleatórios.
A conclusão é que, para o modelo usado, a incapacidade de distinguir dados de alta entropia de dados aleatórios pode explicar apenas 1/3 dos erros.
Isso sugere que, para pesquisas futuras sobre classificação de fragmentos de arquivos, a busca por um procedimento para rotular automaticamente estruturas internas de arquivos pode ser uma abordagem promissora.

\end{resumo}



