\begin{abstract}{computação forense, \textit{data carving}, aprendizado de máquina, redes neurais, \textit{long short-term memory}, camadas convolucionais, classificação de fragmentos de arquivo, dados de alta entropia}

Dados de alta entropia, freqüentemente encontrados em imagens, vídeos e arquivos compactados, podem ser confundidos como dados aleatórios.
Esta pesquisa explora até que ponto a presença de dados de alta entropia pode explicar erros de classificação de fragmentos de arquivo. A pesquisa está dividida em três partes. Na primeira, a acurácia de alguns tipos de redes neurais é comparada, classificando os fragmentos de arquivo extraídos do conjunto de dados Govdocs1 \cite{garfinkel_bringing_2009}, usando extensões de arquivo como classes. Em seguida, a influência do número de classes na acurácia dos modelos resultantes é explorada. Finalmente, um experimento é construído para medir, para uma determinada arquitetura de rede neural, o número de fragmentos que possuem estruturas reconhecíveis para cada tipo de arquivo. A conclusão é que, para o modelo usado, que classifica fragmentos de 28 tipos de arquivos, a interpretação incorreta de dados de alta entropia como dados aleatórios pode apenas explicar um erro de acurácia de 16,5\%. Como o erro total na acurácia é de 46\% \todo{revisar posteriormente} para 28 classes, 29,5\% das amostras são classificadas incorretamente por outros motivos. Isso sugere que, para pesquisas futuras sobre classificação de fragmentos de arquivos, a busca por um procedimento para rotular automaticamente estruturas internasde arquivos pode ser uma abordagem promissora.

\end{abstract}