\begin{resumo}{computação forense, \textit{data carving}, aprendizado de máquina, redes neurais, LSTM, camadas convolucionais, classificação de fragmentos de arquivo, entropia}

Este trabalho está dividido em três partes.
Na primeira, os valores de acurácia de alguns tipos de redes neurais são comparados, classificando fragmentos de arquivo extraídos do \textit{dataset} Govdocs1 \cite {garfinkel_bringing_2009}, usando suas extensões de arquivo como classes.
Treinar modelos de três camadas com os mesmos tipos de arquivo usados em outros trabalhos resultou em valores de acurácia semelhantes aos deles. Isso sugeriu que a composição do conjunto de dados teve um papel mais relevante nos resultados do que os detalhes individuais da arquitetura dos modelos.
Na segunda parte, foi explorada a influência do número de classes na acurácia. Observou-se que um aumento no número de classes tem a tendência de diminuir a acurácia e aumentar a precisão. Mas o número de classes isoladamente tem uma importância menor que o tipo de extensão selecionado: alguns tipos de arquivo, quando incluídos no experimento, têm um impacto negativo muito maior do que outros, especialmente aqueles que usam compactação ou contêm imagens.
Finalmente, um experimento foi planejado para medir, para uma determinada arquitetura de rede neural, o número de fragmentos que possuem estruturas reconhecíveis para cada tipo de arquivo. A conclusão é que, para o modelo usado, a dificuldade de distinguir dados de alta entropia de dados aleatórios pode explicar apenas cerca de 1/3 dos erros.
Isso sugere que, para pesquisas futuras sobre classificação de fragmentos de arquivos, a busca por um procedimento para automaticamente criar categorias de estruturas internas de arquivos pode ser uma abordagem promissora.

\end{resumo}