
%from pep 2.1, paragrafos 12, 13
Amirami et al.  \cite{amirani_new_2008} appear to be the first 
to provide a viable alternative to classical data carving tools using a neural network approach. Two previous works were found using neural networks with data carving related goals, by Dunhan et al. \cite{dunham_classifying_2005} and Harris \cite{harris_using_2007}, but the first worked with encrypted files only and the second did not achieve good results.

%from pep 2.1, paragrafos 14
Amirami et al.  \cite{amirani_new_2008} used Principal Component Analysis (PCA) as input for a 5 layer feed-forward auto-associative unsupervised neural network to do feature extraction and a 3 layer Multi Layer Perceptron (MLP) to perform classification. They used a similar approach in 2013 \cite{amirani_feature-based_2013}.
\sigla{PCA}{Principal Component Analysis}
\sigla{MLP}{Multi Layer Perceptron}

%from pep 2.1, paragrafos 16,17,18
Other works were found applying neural networks to perform data carving, also using some form of dimensionality reduction as PCA. Ahmed et al. \cite{ahmed_content-based_2010}\cite{ahmed_fast_2011} used byte frequency, 
Penrose et al. \cite{penrose_approaches_2013} used compression rate,
and Maslim et al. \cite{maslim_distributed_2014} used PCA, as did Amirami et al.  \cite{amirani_new_2008}.

%from pep 2.1, paragrafos 15
Xu and Dong \cite{xu_reassembling_2009} used a neural network as a cluster reassembling technique for JPEG image fragments.

%from pep 2.1, paragrafos 19
Hiester \cite{hiester_file_2018} apparently was the first to not resort to dimensionality reduction and also the first to utilize a LSTM network to perform file fragment classification. He compared results using three types of neural networks: feedforward, convolutional, and long short-term memory. The goal was to classify the data type of individual sectors (512 bytes), considering four file types: CSV, XML, JPG and GIF.

% Table \ref{tab:datacarvingstudies} summarizes the  machine learning techniques used in each data carving study.
% \begin{table}[!ht]
\caption{Data carving studies using machine learning}
\label{tab:datacarvingstudies}
\begin{tabular}{l|c|c|c|c}

                                & \multicolumn{4}{c}{Technique} \\ \hline
Study                                     & SVM    & kNN    & NN   & ELM   \\ \hline
\hline
Dunham et al. \cite{dunham_classifying_2005}       &        &        & x    &       \\ \hline
Harris \cite{harris_using_2007}             &        &        & x    &       \\ \hline
Amirani et al. \cite{amirani_new_2008}              &        &        & x    &       \\ \hline
Xu and Dong \cite{xu_reassembling_2009}          &        &        & x    &       \\ \hline
Ahmed et al. \cite{ahmed_content-based_2010}      &        &        & x    &       \\ \hline
Axelsson \cite{axelsson_normalised_2010}      &        & x      &      &       \\ \hline
Conti et al. \cite{conti_automated_2010}          &        & x      &      &       \\ \hline
Ahmed et al. \cite{ahmed_fast_2011}               & x      & x      & x    &       \\ \hline
Gopal et al. \cite{gopal_statistical_2011}        & x      & x      &      &       \\ \hline
Luigi and Stefano \cite{luigi_file_2011}               & x      &        &      &       \\ \hline
Sportiello and Zanero \cite{sportiello_file_2011}          & x      &        &      &       \\ \hline
Fitzgerald et al. \cite{fitzgerald_using_2012}         & x      &        &      &       \\ \hline
Sportiello and Zanero \cite{sportiello_context-based_2012} & x      &        &      &       \\ \hline
Amirani et al. \cite{amirani_feature-based_2013}    & x      &        & x    &       \\ \hline
Beebe et al. \cite{beebe_sceadan:_2013}           & x      &        &      &       \\ \hline
Penrose et al. \cite{penrose_approaches_2013}       &        &        & x    &       \\ \hline
Qiu et al. \cite{qiu_new_2014}                  & x      &        &      &       \\ \hline
Maslim et al. \cite{maslim_distributed_2014}       &        &        & x    &       \\ \hline
Zhang et al. \cite{zhang_svm_2016}                & x      &        &      & x      \\ \hline
Ali et al. \cite{ali_classification_2018}       &        &        &      & x     \\ \hline
Hiester \cite{hiester_file_2018}             &        &        & x    &       \\ %\hline
\end{tabular}
\end{table}