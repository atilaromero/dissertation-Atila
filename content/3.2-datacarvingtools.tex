%from pep 4, paragrafo 2
Several authors reviewed  available data carving tools
\cite{ali_review_2018}
\cite{qiu_new_2014}
\cite{nadeem_ashraf_forensic_2013}
\cite{roux_reconstructing_2008}, 
but the tool listing from Ali \textit{et al.} \cite{ali_review_2018} was found to be the most comprehensive one. Among the listed tools, only Foremost \cite{kendall_foremost_2019}, Scalpel \cite{richard_iii_scalpel:_2005}, and PhotoRec \cite{grenier_photorec_2019} support a wide range of file formats. For example, Photorec supports more than 300 file types.

%from pep 4, paragrafo 1
The available data carving tools generally do not take advantage of the latest techniques that research on the field offers, often still relying on header/footer identification and providing limited reassembling capabilities.
%from pep 2, paragrafo 11
According to Ali \textit{et al.} \cite{ali_review_2018}, artificial intelligence techniques are found to be not fully utilized in this field.

Comparing the accuracy between research papers and available software is difficult because, as they rely on header and footer identification, their performance would be more properly compared to whole file classification instead of file fragment classification, the latter being a harder problem.

% \todo[inline]{table listing tools and their features? not sure, they do not do file fragment classification}

